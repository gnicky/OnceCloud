\section{简介}
网络配置工具提供针对基本网络组件的配置和管理能力,同时适用于基础网络和虚拟网络。
在后续开发过程中还会增加针对特定网络组件的性能数据监视,这些数据可能无法很方便地从Xen Hypervisor中得到,如经过防火墙各规则的具体流量等)。

目前网络配置工具了基于轻量级HTTP服务的RESTful Web Service,并提供了配套的Python API。RESTful Web Service遵循REST的基本规则\cite{fielding2000architectural},具体语义如下:

\begin{itemize}
\item GET请求:获取资源列表
\item POST请求:创建或修改资源
\item PUT请求:替换资源
\item DELETE请求:删除资源
\end{itemize}

具体到各个网络组件的API说明如下。需要说明以下几点:

\begin{itemize}
\item 所有传入参数均以HTTP Header的形式提供。参数名以x-bws开头。
\item API返回响应时响应文本在HTTP Response的正文中提供,响应文本为XML格式。
\item API的调用结果以HTTP Status Code的形式提供,目前会有三种可能的状态码:API成功调用时返回200 OK,参数错误时返回400 Bad Request,其余错误返回500 Internal Server Error。
\item 调用失败时响应正文会包含错误信息,错误信息为XML格式。示例如例程\ref{error-example}所示。
\begin{lstlisting}[caption=错误信息示例,label=error-example,language=XML]
<?xml version="1.0" encoding="UTF-8"?>
<Error>
	Please specify Xxx, Xxx and Xxx.
</Error>
\end{lstlisting}
\end{itemize}

\section{DHCP}
网络配置工具提供了针对DHCP静态绑定的配置功能。

\subsection{GET /DHCP}
说明:获得当前的DHCP配置。

传入参数:无

响应示例:
\begin{lstlisting}[language=XML]
<?xml version="1.0" encoding="UTF-8"?>
<DhcpConfiguration>
	<GlobalConfiguration>
		<Option>
			<Key>ddns-update-style</Key>
			<Value>interim</Value>
		</Option>
		<Option>
			<Key>ignore</Key>
			<Value>client-updates</Value>
		</Option>
	</GlobalConfiguration>
	<SubnetConfiguration>
		<Subnet>
			<SubnetAddress>192.168.118.0</SubnetAddress>
			<Netmask>255.255.255.0</Netmask>
			<Routers>192.168.118.1</Routers>
			<SubnetMask>255.255.255.0</SubnetMask>
			<DomainNameServers>114.114.114.114</DomainNameServers>
			<Range>
				<Start>192.168.118.100</Start>
				<End>192.168.118.200</End>
			</Range>
			<DefaultLeaseTime>21600</DefaultLeaseTime>
			<MaxLeaseTime>43200</MaxLeaseTime>
		</Subnet>
		<Subnet>
			<SubnetAddress>192.168.121.0</SubnetAddress>
			<Netmask>255.255.255.0</Netmask>
			<Routers>192.168.121.1</Routers>
			<SubnetMask>255.255.255.0</SubnetMask>
			<DomainNameServers>114.114.114.114</DomainNameServers>
			<Range>
				<Start>192.168.121.100</Start>
				<End>192.168.121.200</End>
			</Range>
			<DefaultLeaseTime>21600</DefaultLeaseTime>
			<MaxLeaseTime>43200</MaxLeaseTime>
		</Subnet>
	</SubnetConfiguration>
	<HostConfiguration>
		<Host>
			<Name>HostA</Name>
			<HardwareAddress>12:34:56:78:AB:CD</HardwareAddress>
			<IPAddress>192.168.118.21</IPAddress>
		</Host>
		<Host>
			<Name>HostB</Name>
			<HardwareAddress>87:65:43:21:CD:AB</HardwareAddress>
			<IPAddress>192.168.118.22</IPAddress>
		</Host>
		<Host>
			<Name>HostC</Name>
			<HardwareAddress>34:12:56:78:AB:CD</HardwareAddress>
			<IPAddress>192.168.121.21</IPAddress>
		</Host>
		<Host>
			<Name>HostD</Name>
			<HardwareAddress>65:87:43:21:CD:AB</HardwareAddress>
			<IPAddress>192.168.121.22</IPAddress>
		</Host>
	</HostConfiguration>
</DhcpConfiguration>
\end{lstlisting}




\section{NAT}

